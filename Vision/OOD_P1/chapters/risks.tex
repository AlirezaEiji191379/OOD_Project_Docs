‫\فصل{سند ریسک‌ها}
‫\قسمت{ریسک‌های مربوط به محدوده}
‫
‫\زیرقسمت{روشن نبودن محدوده پروژه}
‫\مهم{توضیح} نیازمندی‌ها و جزئیات آنها مشخص نبود و تعریف دقیقی از جزئیات وجود نداشت.
‫\\
‫\مهم{راه‌حل} تعامل بیشتر با مشتری و سایر‌ ذینفعان برای به دست آوردن اطلاعات بیشتر، مطالعه و بررسی سیستم‌های مشابه قاصدک.
‫
‫\زیرقسمت{مبهم بودن نیازمندی‌های مشتری و عدم آگاهی تیم به آن}
‫\مهم{توضیح} از آنجا که مهندسی نیازمندی‌ها مهم‌ترین بخش از مراحل ایجاد سیستم است، تیم برای گام‌های بعدی، به این دانش نیازمند است.
‫\\
‫\مهم{راه حل} فیدبک گرفتن از مشتری در ادامه‌ مسیر و حفظ تعامل با مشتری، تهیه prototypeها به منظور کسب اطلاعات بیشتر از مشتری و دقیق کردن آنها در طول فرآیند ایجاد
‫
‫\قسمت{ریسک‌های مربوط به زمانبندی}
‫\زیرقسمت{تخمین‌ نادرست از میزان زمان لازم برای پایان دادن به هر تسک}
‫\مهم{توضیح} تیم ایجاد به دلیل عدم آگاهی و جدید بودن اعضا با فرآیند، ممکن است در انجام تسک‌ها دچار خطاهایی شوند که می‌تواند شکست کامل پروژه را به همراه داشته باشد.
‫\\
‫\مهم{راه حل} بررسی زمانبندی‌های تیم‌های سال گذشته، توجه به برنامه زمانی مطرح شده از طرف مشتری، اولویت‌بندی نیازمندی‌ها و اختصاص زمان کافی به نیازمندی‌هایی از نظر مشتری در الویت بالا یعنی رده MUST محسوب می‌گردند، تعامل با مشتری به عنوان راه‌حل نهایی برای کسب فرصت بیشتر در صورت امکان.
‫
‫\زیرقسمت{رخدادهای ناگهانی در طول اجرای پروژه}
‫\مهم{توضیح} ممکن است تیم به رخدادهای ناگهانی بر بخورد که زمانبندی را تحت تاثیر قرار دهد، مانند شرایط ناپایدار جامعه و یا کوییزها یا امتحانات ناگهانی از طرف سایر دروس دانشکده.
‫\\
‫\مهم{راه‌حل} تیم به جمع‌بندی دقیقی برای این مسئله نداشت اما پیشنهاداتی مانند بررسی میزان اهمیت کوییز یا امتحان مطرح شده و در صورت امکان جابه‌جایی یا نادیده‌ گرفتن آن به منظور انجام فعالیت‌های پروژه و یا بررسی میزان حاد بودن اوضاع و جدی بودن تصمیمات و تلاش برای حفظ برنامه‌ پروژه مطرح شد و اعضای تیم بر روی آن توافق کردند.
‫
‫\قسمت{ریسک‌های مربوط به منابع انسانی}
‫\زیرقسمت{حذف درس توسط یکی از اعضا}
‫\مهم{توضیح} ممکن بود یکی از اعضا به دلیل شرایط درس و ترم شخصی خود تصمیم به حذف درس گرفته باشد.
‫\\
‫\مهم{راه‌حل} اعضای تیم بر روی عدم حذف درس توافق کردند.
‫
‫\زیرقسمت{مهاجرت اعضا}
‫\مهم{توضیح} ممکن است یکی از اعضا به دلیل مهاجرت از جایی به بعد نتواند با تیم همراه باشد و تیم را با مشکلات جدی رو‌به‌رو کند.
‫\\
‫\مهم{راه‌حل} راه‌حل قطعی و محکمی برای رفع این مسئله وجود نداشت اما تیم تصمیم گرفته است که کارها را به صورت دوبه‌دو تقسیم کند تا در صورت خروج یکی از اعضا کاری از دست نرود، همچنین امکانی را برای همکاری از راه دور برقرار سازد مانند ارتباط اعضا از طریق بسترهای ارتباط‌ جمعی غیر ایرانی، استفاده از ابزار مدیریت منابع سیستم برای گردهمآروی تمام مستندات و سورس کدها و به‌روز نگه‌داشتن آن‌ها.
‫
‫\قسمت{ریسک‌های مربوط به سناریوهای  سیستم}
‫\زیرقسمت{از دست رفتن اطلاعات مالی کاربران}
‫\مهم{توضیح} اطلاعات مالی کاربران برای سیستم و کارفرما ضروری است که ریسک زیادی را به سیستم اعمال می‌کند.
‫\\
‫\مهم{راه‌حل} در طرح اولیه سیستم، یک زیرسیستم برای مدیریت مسائل مالی در نظر گرفته شد تا در گام طراحی  و پیاده‌سازی به کمک مکانیسم‌های مربوطه تا جای ممکن از وقوع آن جلوگیری شود.
‫
‫\زیرقسمت{مشکلات مربوط به درگاه پرداخت}
‫\مهم{توضیح} عدم امکان برقراری با درگاه پرداخت برای جابه‌جایی مبالغ مربوطه از دیر مسائل جدی سیستم است.
‫\\
‫\مهم{راه‌حل} برای کاهش ریسک، prototype برای سنجش مشکلات مربوطه طراحی شد تا تصمیمات لازم برای آن در گام‌های بعدی را راحت‌تر کند.
‫
‫\زیرقسمت{ مسائل مربوط به نگهداری محتوا و preview}
‫\مهم{توضیح} امکان استفاده از محتوا برای کاربران و نگهداری آن که مکانیسم اصلی سیستم است و عدم سنجش ریسک آن می‌تواند به شکست کلی پروژه منجر شود.
‫\\
‫\مهم{راه‌حل} بررسی راه‌حل‌های موجود برای حل این مشکل در سیستم‌های مشابه و سنجش امکان به‌کارگیری آنها به کمک prototypeها برای رسیدن به یک نتیجه قابل قبول به منظور کاهش ریسک و افزایش امکان‌پذیری پروژه.
‫
‫\زیرقسمت{امکان استریم کردن محتواهای ویدیویی}
‫\مهم{توضیح} این نیازمندی جزو نیازمندی‌های اصلی سیستم نیست اما در صورت تصمیم مشتری برای درنظرگرفتن آن، ریسک انجام آن که به منابع متنوعی نیاز دارد لازم است سنجیده شود.
‫\\
‫\مهم{راه‌حل} تیم در این مرحله تصمیم گرفته است از بررسی آن صرف نظر کند و بعد از تعامل بیشتر با مشتری به سنجش میزان ریسک آن اقدام کند.
‫
‫\زیرقسمت{مسائل مربوط به نیازمندی‌های امنیتی}
‫\مهم{توضیح} محرمانگی اطلاعات و احراز اصالت اعضا با توجه به اینکه سیستم دارای محتواهای premium و اعضای اشتراکی می‌باشد برای مشتری و تیم حائز اهمیت است.
‫\\
‫\مهم{راه‌حل} تیم تصمیم گرفته است تا با جدا کردن دغدغه‌های مربوط به مدیریت کاربران به عنوان به یک زیرسیستم اصلی، مکانیسم‌های لازم را در گام‌های طراحی و پیاده‌سازی به کارگیرد و ریسک مربوطه را کاهش دهد.
‫
‫\قسمت{ریسک‌های فنی}
‫\زیرقسمت{مسائل مربوط به تجهیزات تیم}
‫\مهم{توضیح} خرابی لپتاپ و مشکلات سیستمی که می‌تواند تیم را در مسائلی مانند زمانبندی یا محتواهای فازها، دچار بحران کند.
‫\\
‫\مهم{راه‌حل} نگهداری داده‌ها و حفظ آخرین نسخه تغییرات در فضاهای ابری مورد اعتماد که مشکلات فقدان اطلاعات را برطرف می‌کنند. 
‫
‫
‫
‫
‫