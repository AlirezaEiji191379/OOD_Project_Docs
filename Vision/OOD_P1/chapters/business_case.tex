‫\فصل{جایگاه‌یابی محصول}
‫
‫در این بخش ابتدا مروری بر مسئله و فرصت ارائه راه حل برای آن می‌کنیم و سپس با ارائه تحلیلی کلی از بازار، توجیه اقتصادی این  پروژه را شرح می‌دهیم و در نهایت با بررسی و تخمین هزینه‌ها و مزایای احتمالی، درخواست تایید سرمایه گذاری را برای بخش‌های ارشد و بالاتر سازمان ثبت می‌کنیم.
‫
‫\قسمت{مشکل و فرصت}
‫با گسترش چشمگیر تکنولوژی‌ها و تغییر ناگزیر شیوه‌های ارتباطی بین مردم در سالیان گذشته، شبکه‌های اجتماعی به یکی از مهمترین اکوسیستم‌های ارتباطی، تبلیغاتی و حتی تجارتی تبدیل شده‌اند. یکی از ارکان اصلی پویایی و جذابیت شبکه‌های مجازی سازندگان محتوا هستند که نقش مهمی در جذب و حفظ مخاطبان ایفا می‌کنند، اما اغلب این مشارکت‌ها، برای آنها مزیت مالی‌ای ندارد. این پروژه فرصتی را برای ایجاد پلتفرمی ارائه می دهد که به سازندگان محتوا این امکان را  می‌دهد تا از کار خود کسب درآمد کنند و یک اکوسیستم نوین برای شبکه‌های اجتماعی ارائه می‌دهد که عادلانه‌تر و پایدارتر است.
‫
‫
‫\شروع{لوح}[ht]
‫\تنظیم‌ازوسط
‫
‫\شروع{جدول}{|p{3cm}|p{8cm}|}
‫\خط‌پر 
‫مشکل&  سازندگان محتوا در شبکه‌های مجازی، امکان کسب درآمد به صورت ساختارمند و مشخص در این شبکه‌ها را ندارند.\\
‫\خط‌پر
‫تاثیر می‌گذارد روی&  میزان رغبت سازندگان محتوا برای صرف وقت و ساخت محتوای جدید \\
‫\خط‌پر
‫که در نتیجه آن&  کمتر به فعالیت و ساخت محتوا در این شبکه‌ها می‌پردازند و به تبع آن، شبکه مجازی، پویایی خود و نهایتا مخاطبین خود را از دست می‌دهد\\
‫\خط‌پر
‫یک راه حل موفق می‌تواند&  یک شبکه اجتماعی تحت وب باشد که در آن هر فرد این امکان را داشته باشد که حق اشتراک استفاده از محتوای خود را قیمت گذاری کند و از فروش این حق اشتراک‌ها به کاربران، کسب درآمد کند.\\
‫
‫‫
‫\خط‌پر
‫\پایان{جدول}
‫‫\شرح{شرح مشکل و فرصت}
‫\برچسب{جدول:شرح مشکل و فرصت}
‫\پایان{لوح}
‫
‫
‫
‫\قسمت{تحلیل بازار}
‫تعداد تولیدکنندگان محتوا، اینفلوئنسرها یا سایر افرادی که از رسانه های اجتماعی کسب درآمد می کنند به طور مداوم در حال افزایش است. محبوب ترین شبکه های اجتماعی برای کسب درآمد عبارتند از:
‫
‫
‫\شروع{فقرات}
‫\فقره 	
‫\مهم{اینستاگرام} طبق گزارش emarketer.com، اینستاگرام به یک پلتفرم اصلی برای اینفلوئنسر مارکتینگ تبدیل شده است و این شرکت featureهای متفاوتی را برای پشتیبانی از مارکتینگ در محصول خود ارائه کرده است. این ویژگی‌ها شامل Instagram shopping است که به کاربران امکان می‌دهد محصولات را مستقیماً از پست‌ها خریداری کنند و یا Instagram reels که می‌توان از آن برای ایجاد محتوای ویدیویی کوتاه با قابلیت کسب درآمد استفاده کرد. کاربران همچنین می‌توانند از story یا postهای تبلیغاتی برای کسب درآمد استفاده کنند.
‫
‫\فقره 	
‫\مهم{یوتیوب} یوتیوب یکی از محبوب‌ترین پلتفرم‌ها برای تولیدکنندگان محتوای ویدیویی است و راه‌های مختلفی برای کسب درآمد از محتوا از جمله تبلیغات، اسپانسرشیپ و فروش کالا ارائه می‌دهد. به گزارش codefuel.com، یوتیوب دومین وب سایت پربازدید در جهان است و بیش از 2 میلیارد کاربر فعال ماهانه دارد..
‫
‫\فقره 
‫\مهم{TikTok} 
‫این پلتفرم به سرعت به یکی از محبوب ترین پلتفرم های رسانه های اجتماعی، به ویژه در میان کاربران جوان تبدیل شده است. این برنامه چندین راه برای کسب درآمد از محتوا، از جمله تبلیغات، اسپانسرشیپ و پخش زنده ارائه می دهد. به گفته beverlyboy.com، TikTok بیش از 1 میلیارد کاربر فعال ماهانه دارد.
‫
‫
‫\پایان{فقرات}
‫
‫ارزش خالص سهام شرکت های شبکه های اجتماعی به طور مداوم در حال تغییر است، اما تا ماه می 2023، برخی از با ارزش ترین شرکت های رسانه های اجتماعی و دارایی خالص آنها عبارتند از:
‫
‫\شروع{فقرات}
‫
‫\فقره 	
‫\مهم {فیس بوک (متا)} طبق beverlyboy.com، در ماه می 2023، دارایی خالص فیس بوک بیش از 1 تریلیون دلار بود.
‫
‫\فقره 	
‫\مهم{توییتر} طبق گزارش barnesandnoble.com، در ماه می 2023، دارایی خالص توییتر تقریباً 50 میلیارد دلار بود
‫
‫\فقره
‫\مهم{
‫(Snapchat)
‫Inc.
‫Snap
‫}
‫طبق codefuel.com، در ماه می 2023، دارایی خالص Snap Inc بیش از 100 میلیارد دلار بود.
‫
‫\پایان{فقرات}
‫
‫پیش‌بینی دقیق میزان رشد سهام شرکت‌های شبکه‌های اجتماعی در سال‌های آینده دشوار است، اما طبق گزارش codefuel.com، انتظار می‌رود بازار جهانی رسانه‌های اجتماعی با نرخ رشد مرکب سالانه (CAGR) 8/9 درصد بین سال‌های 2021 تا 2026 رشد کند. این رشد تحت تأثیر عواملی مانند افزایش نفوذ اینترنت، محبوبیت روزافزون رسانه‌های اجتماعی در میان گروه‌های سنی بالاتر و افزایش بازاریابی تأثیرگذار است. علاوه بر این، از آنجایی که شرکت های رسانه های اجتماعی به معرفی ویژگی های جدید برای حمایت از تجارت الکترونیک و سایر استراتژی های کسب درآمد ادامه می دهند، ارزش خالص آنها احتمالاً افزایش می یابد. بنابراین به طور کلی می‌توان گفت بر اساس داده‌ها، روند رو به رشدی به سمت پلت‌فرم‌های رسانه‌های اجتماعی مبتنی بر اشتراک وجود دارد که به سازندگان محتوا اجازه می‌دهد به‌جای تکیه بر مدل‌های درآمد مبتنی بر تبلیغات، مستقیماً از کاربران درآمد کسب کنند. 
‫
‫
‫\قسمت{توجیه اقتصادی}
‫
‫با توجه به این روندها، یک پلتفرم رسانه اجتماعی مبتنی بر وب که به سازندگان محتوا اجازه می‌دهد تا اشتراک کانال‌های خود را بفروشند، می‌تواند یک فرصت تجاری مناسب باشد. برخی از توجیهات اقتصادی این پروژه عبارتند از:
‫
‫
‫\شروع{فقرات}
‫
‫\فقره 	
‫\مهم{بازار در حال رشد} طبق گزارش businessnewsdaily.com، انتظار می‌رود بازار جهانی رسانه‌های اجتماعی با نرخ رشد مرکب سالانه (CAGR) 8/9 درصد بین سال‌های 2021 تا 2026 رشد کند. این رشد تحت تأثیر عواملی مانند افزایش نفوذ اینترنت، محبوبیت فزاینده رسانه های اجتماعی در میان گروه های سنی بالاتر و افزایش بازاریابی تأثیرگذار است.
‫
‫\فقره	
‫\مهم{درآمد پایدار} یک مدل مبتنی بر فروش اشتراک، منبع درآمد پایدارتری را برای سازندگان محتوا فراهم می‌کند، زیرا در این صورت آنها صرفاً به درآمد تبلیغات وابسته نخواهند بود، که منبع غیرقابل پیش‌بینی‌ای است و ممکن است نوسانات زیادی داشته باشد.
‫
‫\فقره 	
‫\مهم{ارتباط مستقیم با مخاطب} این مدل از شبکه‌های مجازی به سازندگان محتوا اجازه می‌دهد تا رابطه مستقیمی با مخاطبان خود ایجاد کنند که می‌تواند منجر به جذب طرفداران وفادارتر و متعهدتر شود. این همچنین می‌تواند به کاهش spam و مسمومیت محتوا در پلتفرم کمک کند، زیرا کاربرانی که برای محتوا هزینه می‌کنند، بیشتر روی تجربه خود سرمایه‌گذاری می‌کنند و کمتر درگیر رفتار منفی می‌شوند.
‫
‫\فقره 
‫\مهم{پتانسیل رشد} با رشد پلتفرم و پیوستن تولیدکنندگان محتوا، پتانسیل گسترش این پلتفرم در حوزه‌های دیگر مانند تجارت الکترونیک یا رویدادهای زنده وجود دارد که می‌تواند جریان‌های درآمد بیشتری را فراهم کند.
‫
‫\پایان{فقرات}
‫
‫
‫\قسمت{تجزیه و تحلیل هزینه‌ها‌ و مزایا}
‫
‫هزینه‌ها:
‫\شروع{فقرات}
‫
‫\فقره توسعه و طراحی پلتفرم
‫\فقره	هزینه های زیرساخت و میزبانی
‫\فقره بازاریابی و تبلیغات
‫\فقره هزینه های قانونی و انطباق
‫\فقره	نگهداری و پشتیبانی مداوم
‫
‫\پایان{فقرات}
‫
‫مزایا:
‫\شروع{فقرات}
‫
‫\فقره	درآمد حاصل از اشتراک های پولی و تبلیغات
‫\فقره	افزایش ماندگاری تولیدکنندگان محتوا در شبکه
‫\فقره	افزایش تعامل کاربران در محیط
‫\فقره	پتانسیل برای مشارکت و همکاری با سایر مشاغلم
‫
‫\پایان{فقرات}
‫
‫
‫\قسمت{پیش بینی های مالی}
‫
‫با فرض اینکه این پروژه را در سطح یک پروژه متوسط نگهداریم و با تیمی متشکل از 4 توسعه دهنده در زمان تخمینی 6 تا 12 ماه پروژه را کامل کنیم، بسته به نرخ ساعتی توسعه تیم و scope پروژه، هزینه ممکن است از 100000 تا 500000 دلار یا بیشتر متغیر باشد. هزینه‌های پروژه به تفکیک شامل موارد زیر است:
‫
‫
‫\شروع{فقرات}
‫
‫\فقره 
‫\مهم{تیم توسعه} هزینه استخدام توسعه دهندگان، طراحان، آزمایش کنندگان و مدیران پروژه در طول مدت پروژه. این هزینه به نرخ ساعتی اعضای تیم و تعداد ساعات کار آنها در پروژه بستگی دارد.
‫\فقره 
‫\مهم{هزینه‌های زیرساخت} هزینه سرورها، host و سایر زیرساخت‌های مورد نیاز برای اجرای پلتفرم. این هزینه به مقیاس پلتفرم و ارائه دهنده سرویس ابری استفاده شده بستگی دارد.
‫\فقره 
‫\مهم{ابزارها و خدمات شخص ثالث} هزینه استفاده از ابزارها و خدمات شخص ثالث مانند درگاه‌های پرداخت، ابزارهای تجزیه و تحلیل و خدمات ایمیل را شامل می‌شود. این هزینه به استفاده و قیمت ابزار و خدمات بستگی دارد.
‫\فقره 
‫\مهم{بازاریابی و تبلیغات} هزینه بازاریابی و ارتقاء پلتفرم برای جذب کاربران و تولیدکنندگان محتوا. این هزینه به استراتژی بازاریابی مورد استفاده، مانند تبلیغات رسانه های اجتماعی، بازاریابی تأثیرگذار یا بازاریابی محتوا بستگی دارد.
‫\فقره	
‫\مهم{تعمیر و نگهداری و پشتیبانی} هزینه نگهداری و پشتیبانی پلتفرم پس از راه اندازی، شامل رفع اشکال، بروزرسانی و پشتیبانی مشتری. این هزینه به اندازه کاربران و پیچیدگی پلتفرم بستگی دارد.
‫
‫\پایان{فقرات}
‫
‫
‫
‫\قسمت{نتیجه گیری و درخواست تایید}
‫
‫این پروژه، پتانسیل‌های تجاری زیادی را با ایجاد یک شبکه اجتماعی با قابلیت کسب درآمد برای تولیدکنندگان فعال دارد که به نفع سازندگان محتوا، مدیران کانال و کاربران نهایی است. این پروژه از نظر فنی و اقتصادی امکان پذیر است و برنامه ای مشخص برای مقابله با ریسک‌ها و چالش‌ها دارد. برای ادامه توسعه و راه‌اندازی این پلتفرم نوآورانه، به تایید مدیریت ارشد و سرمایه‌گذاران نیاز داریم.