‫\فصل{بررسی امکان پذیری‌ها}
‫\قسمت{بررسی امکان‌پذیری اقتصادی}
‫این پلتفرم پتانسیل ایجاد درآمد از طریق اشتراک های پولی، تبلیغات و مشارکت با سایر مشاغل را دارد. هدف این پروژه فراهم کردن بستری برای تولیدکنندگان محتوا برای کسب درآمد از محتوایشان است که می‌تواند برای پلتفرم درآمد ایجاد کند. این پلتفرم می‌تواند با دریافت کمیسیون از درآمد تولیدکنندگان محتوا، دریافت هزینه اشتراک از کاربران یا نمایش تبلیغات در پلتفرم، درآمد کسب کند. تحقیق و تحلیل بازار می تواند به ارزیابی سودآوری بالقوه پروژه کمک کند. با توجه به تحلیل بازار که در قسمت‌های پیشین آورده شد، به نظر می رسد این پروژه از نظر اقتصادی امکان پذیر باشد.
‫\قسمت{بررسی امکان‌پذیری فنی}
‫با توجه به سامانه‌های موجود، توسعه یک شبکه اجتماعی مبتنی بر وب با ویژگی های ثبت نام کاربر، اشتراک گذاری محتوا و کسب درآمد از نظر فنی با استفاده از چارچوب‌ها و فناوری‌های مدرن توسعه وب امکان پذیر است. این پروژه به یک پلتفرم مبتنی بر وب نیاز دارد که بتواند محتوای تولید شده توسط کاربر را میزبانی کند، پرداخت ها را پردازش کند و حساب های کاربری را مدیریت کند. این ویژگی ها به راحتی از طریق چارچوب‌ها و ابزارهای مختلف توسعه وب در دسترس هستند. تیم فنی باید قادر به توسعه و نگهداری پروژه باشد. مقیاس پذیری، قابلیت اطمینان و نگهداری پروژه را می توان با پیروی از بهترین شیوه‌ها و استفاده از ابزارها و فناوری‌های استاندارد صنعتی تضمین کرد. بنابراین، این پروژه از نظر فنی امکان پذیر است.
‫
‫\قسمت{بررسی امکان‌پذیری حقوقی}
‫این پلتفرم باید با قوانین حفاظت از داده‌ها و حریم خصوصی و همچنین قوانین کپی رایت و مالکیت معنوی مطابقت داشته باشد. همچنین شرایط و ضوابط پروژه باید شفاف و مطابق با قوانین مربوطه باشد. برای اطمینان از انطباق، تمامی موارد با منشور‌های حقوقی در موجود بررسی خواهد شد.
‫
‫\قسمت{بررسی امکان‌پذیری برنامه‌ریزی}
‫با توجه به چرخه‌ها و فاز‌های تعریف شده و میزان زمان تخمین زده شده برای انجام هر کار، پروژه در زمان معین شده قابل انجام است
‫
‫\قسمت{بررسی امکان‌پذیری عملیاتی}
‫این پلتفرم به نگهداری مداوم، به روزرسانی و پشتیبانی مشتری نیاز دارد. یک تیم اختصاصی برای مدیریت این وظایف ایجاد وجود دارد.