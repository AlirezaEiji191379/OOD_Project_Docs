‫\فصل{مروری به محصول}
‫\قسمت{چشم‌انداز محصول}
‫قاصدک یک سیستم مستقل است و بخشی از یک سیستم بزرگ‌تر نیست. این شبکه اجتماعی بدین صورت است که شامل تعدادی کانال است که در آن محتواهای متن، عکس، ویديو و فایل صوتی گذاشته می‌شود. در کانال توسط صاحب کانال یا صاحب کانال و مدیر کانال محتوا تولید می‌شود. شرایط استفاده از کانال به تعریف صاحب کانال وابسته است می‌تواند استفاده از محتویات کانال را به صورت رایگان امکان‌پذیر کند. همچنین می‌تواند حق عضویت یک، سه، شش، دوازده ماهه تعریف کند که در این صورت کاربرانی که اشتراک را تهیه نکنند تنها قادر به مشاهده عنوان و خلاصه از آن محتوا هستند. از آنجا که تولید‌کنندگان محتوا در هر شبکه نقش مهمی در پویایی و جذب و نگهداری مخاطب در آن شبکه ایفا می‌کنند اما از نظر اقتصادی نفعی از این مسئله نمی‌برند به همین دلیل شبکه اجتماعی قاصدک حق عضویت را برای کاربران تعریف کرده است تا تولیدکنندگان محتوا بتوانند با تولید و به اشتراک‌گذاری محتوا درآمد کسب کنند.
‫
‫\قسمت{خلاصه کاربرد‌های محصول}
‫
‫\شروع{لوح}[ht]
‫\تنظیم‌ازوسط

‫\شروع{جدول}{|c|c|}
‫\خط‌پر 
‫\سیاه نفع کاربر & \سیاه قابلبت‌های پشتیبانی شده \\ 
‫\خط‌پر
‫مطلع‌ کردن & به اشتراک‌گذاری محتواهای متن، عکس، ویديو و فایل صوتی در کانال برای کاربران\\ 

‫\خط‌پر
‫\پایان{جدول}
‫‫\شرح{خلاصه امکانات محصول برای کاربران}
‫
‫\برچسب{جدول:خلاصه امکانات محصول برای کاربران}
‫\پایان{لوح}
‫
‫\شروع{لوح}[ht]
‫\تنظیم‌ازوسط

‫\شروع{جدول}{|c|c|}
‫\خط‌پر 
‫\سیاه نفع تولید‌کننده محتوا & \سیاه قابلبت‌های پشتیبانی شده \\ 
‫\خط‌پر
‫کسب در‌آمد & تعیین حق اشتراک به صورت یک، سه، شش و دوازده ماهه\\ 

‫\خط‌پر
‫\پایان{جدول}
‫‫‫\شرح{خلاصه امکانات محصول برای کاربران}
‫
‫\برچسب{جدول:خلاصه امکانات محصول برای تولیدکنندگان محتوا}
‫\پایان{لوح}
‫
‫
‫
‫
‫
‫\قسمت{ذینفعان}
‫
‫\شروع{لوح}[ht]
‫\تنظیم‌ازوسط
‫
‫\شروع{جدول}{|p{2cm}|p{5cm}|p{8cm}|}
‫\خط‌پر 
‫\سیاه نام & \سیاه شرح & \سیاه مسئولیت‌ها \\ 
‫\خط‌پر
‫‌ مدیر پروژه & نیازمندی‌های دینفعان را به تیم منتقل می‌کنند و کار تیم را به ذینفعان منتقل می‌کنند & متخصص دامنه مسئله است و مسئول برنامه ریزی، سازماندهی و کنترل فعالیت‌ها مطابق با جدول زمانی تعیین شده است. او باید تمام اطلاعات لازم در مورد پروژه، از جمله الزامات، محدوده و اهداف آن را بداند. مدیر پروژه با ذینفعان خارج از پروژه ارتباط برقرار می کند\\ 
‫\خط‌پر
‫‌ تیم توسعه & محصول را برای ذینفعان آماده می‌کنند & مسئولیت ساخت بنیان کد پلتفرم، پیاده سازی ویژگی‌های آن و آزمایش عملکرد آن را دارند. مسئولیت‌های آن‌ها شامل نوشتن کد تمیز و قابل نگهداری، همکاری با سایر توسعه‌دهندگان و ذینفعان، و عیب‌یابی مشکلاتی است که در طول توسعه ایجاد می‌شود\\ 
‫
‫\خط‌پر
‫سرمایه گذار & منابع مالی مورد نیاز پروژه را تامین می‌کنند & مسئول تامین منابع و پشتیبانی از پروژه است. آنها در ازای سهام یا سایر اشکال مالکیت، بودجه پروژه را تأمین می کنند. مسئولیت آنها حمایت مالی و نظارت بر پروژه با هدف دستیابی به بازگشت سرمایه است\\ 
‫\خط‌پر
‫
‫\پایان{جدول}
‫\پایان{لوح}
‫
‫\شروع{لوح}[ht]
‫\تنظیم‌ازوسط
‫
‫\شروع{جدول}{|p{2cm}|p{5cm}|p{8cm}|}
‫ 
‫\خط‌پر
‫
‫
‫‫‌تولیدکنندگان محتوا & محتوای شبکه اجتماعی را تولید می‌کنند &  این افراد محتوا را در پلتفرم ایجاد و به اشتراک می گذارند و از آن درآمد کسب می کنند. آنها مسئول ایجاد محتوای باکیفیت هستند که مخاطبان خود را درگیر می کند و آنها را تشویق به خرید اشتراک می کند\\ 
‫
‫\خط‌پر
‫صاحبان کانال & کانال‌های داخل شبکه را ایجاد و نگهداری می‌کنند & کانال‌هایی را روی پلتفرم ایجاد می کنند و اشتراک ها را به کاربران می فروشند. آنها از به اشتراک گذاری مطالب در کانال خود درآمد کسب می کنند. آنها مسئول مدیریت تولید محتوای با کیفیت بالا و تبلیغ کانال خود برای جذب مشترک هستند\\ 
‫
‫\خط‌پر
‫کاربران & از قاصدک برای دریافت و تماشای محتوا استفاده می‌کنند & -\\
‫\خط‌پر
‫
‫\پایان{جدول}
‫‫\شرح{ذینفعان}
‫\برچسب{جدول:ذینفعان}
‫\پایان{لوح}
‫\FloatBarrier
‫\شروع{فقرات}
‫\فقره  
‫\مهم{سازندگان محتوا} در این سامانه، سازندگان محتوا می‌توانند با فروش حق اشتراک کانال‌های خود به کاربران، کسب درآمد کنند
‫
‫\فقره  
‫\مهم{تیم توسعه پلتفرم} تیم توسعه پلتفرم می‌تواند درصد معینی از میزان فروش هر فرد را به عنوان حق کمیسیون برای خود بردارد. این تیم همچنین امکان فروش محصول به شرکت‌های دیگر را نیز دارند. همچنین امکان اجرای برنامه‌های تبلیغاتی، اقتصادی و برندینگ نیز وجود دارد که به طرق مختلف، می‌تواند سودآوری داشته باشد.
‫
‫\فقره  
‫\مهم{سرمایه گذاران} با سرمایه گذاری روی این محصول، بخشی از سهام محصول و درصد معینی از میزان درآمد سامانه به سرمایه گذار می‌رسد
‫
‫\پایان{فقرات}
‫
‫
‫
‫\قسمت{فرضیات و وابستگی‌ها}
‫
‫\شروع{فقرات}
‫\فقره زبان اصلی قاصدک انگلیسی است. فرض می‌شود افرادی که از این سامانه استفاده می‌کنند تا حدی با آن آشنایی دارند و از آن می‌توانند به راحتی استفاده کنند.
‫
‫\فقره فرض می‌شود که وسیله الکترونیکی کاربر از مرورگر‌ها پشتیبانی می‌کند.
‫
‫\فقره فرض می‌شود هنگام استفاده از شبکه اجتماعی قاصدک وسیله الکترونیکی به شبکه متصل است.
‫
‫\فقره فرض می‌شود کاربری که می‌خواهد از شبکه اجتماعی استفاده کند دارای ایمیل شخصی یا تلفن‌همراه است که می‌تواند از طریق آن ثبت‌نام، احراز هویت و بازیابی رمز عبور انجام دهد.
‫
‫\پایان{فقرات}
‫
‫
‫
‫
‫