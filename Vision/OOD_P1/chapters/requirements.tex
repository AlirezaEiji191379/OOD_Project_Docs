‫\فصل{سند نیازمندی‌ها}
‫
‫در این فصل نیازمندی‌های سیستم در قالب زیرسیستم‌های مختلف بدون هیچ الویت خاصی آورده شده‌ است.
‫\قسمت{نیازمندی‌های وظیفه‌ای}
‫
‫\زیرقسمت{زیرسیستم کاربر}
‫\شروع{شمارش}
‫\فقره کاربران باید بتوانند از طریق ایمیل معتبر در سیستم حساب کاربری بسازند.
‫‫\فقره کاربران باید بتوانند از طریق تلفن‌همراه معتبر در سیستم حساب کاربری بسازند.
‫\فقره در صورت تایید احراز هویت کاربران باید بتوانند وارد سامانه شوند.
‫‫\فقره همچنین هنگام ثبت‌نام باید بتوانند نام و نام‌خانوادگی را که بعدا به عنوان نام کاربری از آن استفاده می‌شود وارد کنند.
‫\فقره
‫کاربران باید بتوانند نام و نام‌خانوادگی خود که به عنوان نام‌کاربری در نظر گرفته می‌شود را تغییر بدهند. 
‫\فقره
‫کاربران باید بتوانند رمز عبور خود را نیز تغییر بدهند. 
‫\فقره
‫اطلاعات کاربری اعم از نام و نام‌خانوادگی، موجودی حساب و اطلاعات درآمدی از کانال‌ها را باید بتوان مشاهده کرد.
‫\فقره امکان حذف حساب کاربری باید امکان‌پذیر باشد.
‫\فقره باید کاربر بتواند از حساب کاربری خود خارج شود.
‫
‫\زیرقسمت{زیرسیستم کانال}
‫\فقره
‫کاربران لیستی از کانال‌هایی را که عضو آن هستند یا به عنوان تولید‌کننده محتوا در آن هستند باید بتوانند مشاهده کنند.
‫\فقره
‫همچنین باید بتوانند با جست‌وجو کردن نام کانال و انتخاب، در آن عضو شوند.
‫\فقره
‫هر فردی باید بتواند کانال بسازد و هنگام ساخت اطلاعاتی نظیر نام، عکس، درصد سود و هزینه پرمیوم یا رایگان بودن آن را وارد کند.
‫
‫\فقره باید صاحب کانال بتواند مدیر برای کانال تعریف کند.
‫\فقره امکان حذف کانال باید وجود داشته باشد.
‫\فقره
‫در کانال، باید صاحب کانال بتواند کاربران را حذف کند.
‫\فقره
‫باید امکان تعریف medialist برای محتوا وجود داشته باشد.
‫\فقره
‫باید صاحب بتواند دسته‌بندی را مشاهده کند.
‫\فقره
‫باید صاحب بتواند دسته‌بندی را ویرایش کند.
‫\فقره
‫باید صاحب بتواند دسته‌بندی را حذف کند.
‫\فقره
‫باید صاحب کانال بتواند لیست تمام مدیران را مشاهده کند.
‫\فقره
‫در کانال باید امکان حذف مدیران کانال برای صاحب وجود داشته باشد.
‫
‫
‫\زیرقسمت{زیرسیستم محتوا}
‫\فقره صاحب یا مدیر کانال باید بتواند در صورتی که کانال نیاز به حق عضویت نداشته باشد، محتوا رایگان در آن قرار دهد.
‫\فقره صاحب یا مدیر کانال باید بتواند در صورتی که کانال نیاز به حق عضویت داشته باشد، محتوا پریمیوم در آن قرار دهد.
‫\فقره اعضا کانال باید بتوانند در سطح محتوا مانند لایک و کامنت تعامل برقرار کنند.
‫\فقره
‫قابلیت جست‌و‌جو در سطح کانال باید وجود داشته باشد. 
‫\فقره
‫قابلیت جست‌و‌جو در سطح کل سیستم باید وجود داشته باشد. 
‫\فقره
‫صاحب یا مدیر کانال می‌تواند در کانال محتواهایی نظیر متن، ویدیو، صدا و عکس را حذف کند. 
‫\فقره
‫صاحب یا مدیر کانال می‌تواند در کانال محتواهایی نظیر متن، ویدیو، صدا و عکس را ویرایش کند.
‫\فقره
‫در صورتی که محتوا کانال رایگان باشد باید تمام محتوا را مشاهده کنند و در غیر این صورت باید بتوانند با خرید حق عضویت امکان مشاهده بیشتر از عنوان و ابتدا محتوا را کسب کنند.
‫
‫
‫\زیرقسمت{زیرسیستم مالی}
‫\فقره کاربران باید دارای کیف پول مجازی باشند.
‫\فقره باید عایدی مالی به حساب صاحبان قاصدک در وقت مقرر واریز شود.
‫\فقره درآمد حاصل از کانال‌های پریمیوم به حساب صاحب کانال و مدیران باید واریز شود.
‫\فقره کاربران باید بتوانند برای مشاهده محتوای پریمیوم اشتراک حق عضویت را خریداری کنند.
‫\زیرقسمت{زیرسیستم پیشنهاد}
‫\فقره یک سیستم توصیه‌گر می‌تواند محتوای مرتبط با کانال یا کانال‌های مرتبط با کانال‌های محبوب کاربر را به کاربر پیشنهاد بدهد.
‫\فقره سیستم ما می‌تواند دارای یک تایم‌لاین برای کانال‌ها باشد.
‫\پایان{شمارش}
‫‫
‫
‫\قسمت{نیازمندی‌های غیروظیفه‌ای}
‫\زیرقسمت{قابلیت اطمینان (Reliability)}
‫تضمین میشود که 99 درصد از مواقع سرورهای نرم افزار در حال کار باشند و به درستی به کار خود ادامه دهند.
‫\زیرقسمت{دسترس‌پذیری (Availability)}
‫تضمین میشود که در 99 درصد مواقع نرم افزار در دسترس کاربران باشد.
‫\زیرقسمت{امنیت (Security)}
‫تضمین می‌شود نرم افزار از حملات معروف مثل sql injection ، xss ، csrf و ...  در امان باشد.
‫تضمین میشود که نرم افزار بتواند اطلاعات 10000 کاربر را نگه‌داری کند و توانایی کنترل و مدیریت 10  درخواست در ثانیه را داشته باشد.
‫\زیرقسمت{قابلیت نگه‌داری (Maintainability)}
‫ تضمین می‌شود که نرم افزار در صورت مواجهه با باگ با توجه به مفاد ذکر شده در قرار داد با مشتری در اسرع وقت مشکل را حل کند.
‫\زیرقسمت{قابلیت استفاده (Usability)}
‫تضمین می‌شود که واسط کاربری نرم‌افزار ما user friendly باشد و کاربر بتواند به راحتی با آن کار کرده و بدون نیاز به راهنما از آن استفاده کند.
‫