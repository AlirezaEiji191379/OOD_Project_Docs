‫
‫\فصل{مقدمه}
‫در این فصل، هر کدام از بخش‌های مستند که شامل چندین مستنداست به طور خلاصه به همراه نقش آنها آورده شده است.
‫\قسمت{اهداف و شرح مستندات}
‫
‫‫\زیرقسمت{مروری بر محصول}
‫به چشم‌اندازی از محصول و کاربرد‌های محصول پرداخته می‌شود تا به طور شفاف ویژگی‌های محصول و کاربرد‌های آن برای ذینفعان مشخص گردد.
‫
‫‫\زیرقسمت{جایگاه‌یابی محصول}
‫با ارائه تحلیلی کلی از بازار، توجیه اقتصادی این  پروژه را شرح می‌دهیم و در نهایت با بررسی و تخمین هزینه‌ها و مزایای احتمالی، درخواست تایید سرمایه گذاری را برای بخش‌های ارشد و بالاتر سازمان ثبت می‌کنیم.
‫
‫‫\زیرقسمت{بررسی امکان‌پذیری‌ها}
‫امکان‌پذیری‌ها از پنج جهت مالی، فنی، برنامه‌ریزی، حقوقی و عملیاتی بررسی می‌شوند.
‫
‫\زیرقسمت{سند الویت‌بندی شده نیازمندی‌ها}
‫استخراج و الویت‌بندی نیازمندی‌‌ها صورت می‌گیرد تا میان اعضای تیم ایجاد و ذینفعان توافقی صورت گیرد و تیم پروژه به شناخت درست از نیازمندی‌ها برسند و ریسک عدم شناخت نیازمندی‌ها به حداقل برسد.
‫
‫‫\زیرقسمت{سند الویت‌بندی شده ریسک‌ها}
‫همچنین استخراج و الویت‌بندی ریسک‌ها نیز صورت می‌گیرد تا درک بهتری از ریسک‌های پروژه ایجاد شود و بتوان برای مدیریت آن‌ها برنامه‌ریزی کرد و بهترین راه‌حل را پیش گرفت. 
‫
‫‫‫\زیرقسمت{سند موارد کاربرد}
‫موارد کاربرد نیز در این مستند شناسایی می‌شوند تا سناریوهای اصلی عملیات که مسائل مربوط به طراحی اصلی را ایجاد می‌کند، شناسایی شوند. به علت محوریت موارد کاربرد در
‫متدلوژی فرایند یکپارچه این سند یکی از مهم‌ترین سند‌های تولید شده است زیرا یک ورودی ضروری در مراحل تحلیل، طراحی و آزمون به حساب می‌آید. موارد کاربرد به گونه‌ای هستند که می‌توانند فارغ از جزییات پیاده‌سازی، برای تعامل با مشتری استفاده شوند.
‫\زیرقسمت{نمونه اولیه واسط کاربری}
‫در این مستند تصاویری از نمونه اولیه واسط کاربری بعد از شناسایی موارد کاربرد و الویت‌بندی، شناسایی می‌شود.
‫\زیرقسمت{نمونه اولیه معماری}
‫در این مستند نمونه اولیه معماری بعد از شناسایی موارد کاربرد و ریسک و الویت‌بندی آن‌ها، شناسایی می‌شود.
‫
‫‫‫\زیرقسمت{برنامه زمان‌بندی شده}
‫در این مستند همچنین یک برنامه اولیه تیم که شامل تسک‌ها، زمان‌بندی تخمینی و مسئولین انجام تسک‌ها است، شناسایی می‌شود تا تیم در جهت درست پیش بروید. این کمک می‌کند تا بهترین و بدترین سناریوهای احتمالی را برای هر فعالیت در پروژه در نظر گرفته و جدول زمانی پروژه را تا حد مشخصی منعطف کرد.
‫
‫‫‫\زیرقسمت{واژه‌نامه}
‫در این بخش توضیحی راجع به واژگان به کار رفته در بقیه مستندات به همراه واژگان مترادف و متشابه آورده شده است. هدف این فصل، به وجود آوردن حوزه معنایی مشترک در مورد دامنه‌ی مسئله میان مشتریان و اعضای تیم ایجاد است تا جلوی سوتفاهم‌های احتمالی در دامنه‌ی مسئله تا حد امکان گرفته شود.
‫
‫\قسمت{گستره مستند}
‫این مستند مربوط به وب‌سایت قاصدک می‌باشد و تیم ایجاد بعد از دریافت توصیف اولیه‌ی پروژه از ذینفعان آن را تهیه کرده است و درصورت نیاز با مشورت ذینفعان در آن تغییر داده می‌شود. این مستند سطح بالا است و برای هماهنگی بین خود اعضای تیم ایجاد و نیز بین اعضای تیم ایجاد و ذینفعان استفاده خواهد شد. بنابراین تنها جزئیاتی آورده شده‌اند که در تصمیم‌گیری حیاتی هستند و تا جای ممكن از تصمیم‌گیری‌های زودهنگام پرهیز شده است.
‫
‫
‫\قسمت{ساختار مستند}
‫این مستند از سیزده فصل (شامل مقدمه) تشكیل شده است.
‫در فصل دوم به چشم‌انداز و کاربرد محصول پرداخته می‌شود.
‫در فصل سوم بر جایگاه‌یابی محصول تمرکز می‌شود.
‫در فصل چهارم به بررسی امکان‌پذیری‌ها از جنبه‌های مختلف پرداخته می‌شوند. 
‫در فصل پنجم نیازمندی‌های محصول شناسایی می‌شود. 
‫در فصل بعدی آن به الویت‌بندی نیازمندی‌ها پرداخته می‌شود. 
‫در فصل هفتم ریسک‌ها شناسایی می‌شود. 
‫در فصل بعدی آن به الویت‌بندی ریسک‌ها پرداخته می‌شود. 
‫در فصل نهم موارد کاربر شناسایی می‌شود و به توضیحات آن پرداخته می‌شود. 
‫در فصل دهم نمونه اولیه واسط کاربری گردآوری می‌شود.
‫در فصل یازدهم نمونه اولیه معماری شناسایی می‌شود.
‫در فصل دوازدهم برنامه زمان‌بندی پروژه شناسایی می‌شود. 
‫در فصل نهایی واژه‌نامه‌ای که شامل واژگان موجود در مستند است، جمع‌آوری شده است.